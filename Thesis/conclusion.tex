\clearpage{\pagestyle{empty}\cleardoublepage}
\chapter*{Conclusion}
\markboth{Conclusion}{Conclusion}
\addcontentsline{toc}{chapter}{Conclusion}

\begin{doublespace}
As highlighted by the two models proposed, Goal Programming indeed may turn out to be a great tool in the hands of the Decision Maker; because of that is largely applied by Supply Chain professionals and academics to model different problems they may face. The same cannot be said when the topics shifts, and from Supply Chain we start dealing with international tax planning problems, as suggested before this may be due to the specific skills that such practitioner has on this field. Because of that is important to renew the call to action to model in a greater way these problems, which in most of the cases boils down to multi-objectives problems that can be solved through Goal Programming.

This thesis wouldn't have been possible without the R project, their packages provided a good link between models built in LINGO and AMPL and solved (in case of the Green Supply Chain model) using a cloud server provider of solvers such as NEOS. The author of this thesis is also grateful to both Supervisor and Co-Supervisor because of the help given to understand the problem better in a mathematical way and also for the provision with tips and tricks in modeling some of the occurrences faced.

Last but not least a special thank goes to the practitioner that helped in shaping the model of international taxation; their help was so much useful as the information they provided me for the case study.    
\end{doublespace}

\clearpage{\pagestyle{empty}\cleardoublepage}
