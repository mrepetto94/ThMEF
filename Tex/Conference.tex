\documentclass{article}

\usepackage{amsmath}
\usepackage{mathtools}
\usepackage{amstext}
\usepackage{cases}
\usepackage[colorinlistoftodos]{todonotes}
\usepackage[long]{optidef}
\usepackage[backend=bibier,
		url=false,
		doi=false,
		isbn=false,
		issn=false]{biblatex}
\addbibresource{ch1-2.bib}

\begin{document}

\title{Tax planning and BEPS: a transportation model for optimal profit allocation}

\author{Marco Repetto}

\date{March 2018}

\maketitle

After the Great Recession, multinational companies found themselves in a very different environment; thriving to survive with competition on one side and higher restriction imposed by countries running financial crises. This led multinationals with a great challenge which consequently boost cost engineering, in an attempt to survive in this new scenario. Another result was a new form of tax planning, most of the time barely legal in order to exploit the fiscal advantages of certain countries\cite{After_tax_hedging_report_2013}. Such problems forced the G20 to launch an inclusive framework on base erosion and profit shifting called "BEPS", bringing together about 100 countries in order to stop such unlawful tax planning. However, even if this action against such problem highlighted in a more specific way what should be considered as aggressive tax planning the border with legitimate cost engineering seems blurrier than ever\cite{feller_three_2017}.
The paper deals with the problem of multinational entity allocation based on three main factors, namely the costs, the tax pressure and the functional characterization of a certain entity. The solution suggested is a Weighted Goal Programming model (WGP) that embeds these limitations imposed by the environment and tries to find an optimal solution that permits the Decision Maker (DM) to be competitive \textit{vis-à-vis} multinational competitors. The results obtained from such model are many times different from the ones put into places by multinationals, which tend to focus on minimizing only the tax base without taking into account other information that may result in significant value added from a tax planning perspective and at the same time lower the tax risk. 
The way to avoid aggressive tax planning, is to take into account any possible index of potential tax risk, thanks to the work of OECD such measures were developed on the new Transfer Pricing guidelines \cite{TPguide2017}, however, this measure has to be considered in a wider perspective with the peculiar features of the firm considered. Because of this numerous trade-offs, pushing from the firm and pulling from the Tax Authorities a systematic approach is needed that takes into account not only one objective but a series of them and that can be suited for the necessity of the DM.
The choice was to pick the Multi-Criteria Decision Analysis; MCDA may be defined as a problem of multiple-objective programming, that differs from a linear one since more objective function is handled; its formulation is the following:
	$$
	Min[f_1(x),f_2(x),...,f_k(x)] \quad i=1,...,k \quad where \quad k\geq2
	$$
	This approach seems to fit better the real world since, in reality, more than one objective is pursued, and most of the time in contrast with one another\cite{greco_multiple_2016}.
A solution to a multi-criteria problem would be optimal if it'd respected the Pareto Efficient assumption, namely that no other feasible solution exists that is at least as good with respect to all objectives and strictly better with respect to at least one objective. Mathematically it means that $\left\{x_1,...x_k\right\}$ is a solution if $\not\exists \left\{x'_1,...x'_k\right\}$
such that:
	\[
	g(f_1(x),f_2(x),...,f_k(x)) \leq g(f_1(x'),f_2(x'),...,f_k(x')) \quad \forall n \quad \in  \left\{1...k\right\}
	\]
However such increase in complexity given by multiple objectives is both the strength and the weakness of such methodology, this is due to the fact that we have to deal with $N$ trade-offs deriving from the objectives we decided to include in our optimization problem. Because of this problem a lot of approaches and specific intelligent algorithms were proposed\cite{Cui2017}.
In particular, in the case under scope there's the necessity to controll completely the trade offs of the DM by controlling the importance given to each and every goal set by the model. Because of this necessity the ultimate choice was to pick the Goal Programming Approach\cite{charnes_optimal_1955}.
The peculiarity of GP is given by its simplicity and it's flexibility\cite{colapinto_multi-criteria_2017}\cite{tamiz_goal_1998}; the idea behind such method is very simple, and is based on distance minimization; this means that at each objective function is associated a deviation variable that has to be minimized in batch with the other deviation variables coming from the other objective functions. Since the goal of the model proposed is to account for direct trade-offs between all unwanted deviational variables the choice was to using weights which are not put a priori, as a result, the Weighted Goal Programming resulted in a more flexible approach. It's mathematical formulation is the following one:

\begin{equation*}
\begin{aligned}
& \underset{n,p}{\text{minimize}}
& & \sum_{q=1}^{Q}(\frac{u_q n_q}{k_q}+\frac{v_q n_q}{k_q}) \\
& \text{subject to}
& & f_q(x)+n_q-p_q=b_q, \; q=1,...Q \\
& & & x\in F \\
& & & n_q,p_q\geq 0, \; q=1,...,Q 
\end{aligned}
\end{equation*}

GP and in particular WGP was used extensively in the field of accounting\cite{aouni_goal_2017}(which is the one in which the model belongs), and turn out to be a very powerful tool in the hand of accountants. However, it's worth mentioning the fact that not micro-field of accounting gained the same attention by both researchers and practitioner and, aside from its popularity, the interest in international tax planning never arose. This may be due to the typical background that such researchers have on the field. In fact, most of the research moved in this field pertain to the legal area, which probably lacks from a point of view of mathematical modeling skills. From the other side, the field of international taxation involves a great set of laws, both soft and hard, that may result in a great effort by the practitioner to model. 
From an extensive review of the field of study under the scope, it’s worth mentioning the work of Merville and Petty\cite{merville_transfer_1978} who tried to model an optimal pricing policy which is indeed useful to set a global tax strategy since the TP core is base on that. However, one major problem of such model is given by the fact that it doesn't take into account the novelties introduced by the recent work of the OECD and the increasing importance of new tools at the disposal of multinational companies such as extensive databases, and newer resource management tools.
Therefore, keeping what proposed by the literature, seems necessary to propose a new type of model which takes into account these new developments and result giving a powerful tool in the hand of the Decision Maker.
The model proposed uses three different sources of data, such data source are: (i)macroeconomic, is the set of data used by DM that came from the national economies; (ii)target enterprise, are the types of data that refer particularly to the firm under scope; (iii)global enterprises data, these data come from other enterprises that are in some way similar to the enterprise under scope, both in terms of business or because of the functions they perform.
Given this set of data the model will have an objective, the minimization of the production costs, at the same time another economical objective will be the minimization of the tax liability defined as Earnings Before Interest and Taxes (EBIT) for the country corporate tax rate, then follows the soft economic objectives that are profit split maintenance in order to avoid the firm to lose its functional characterization and such functional characterization will have to lie within the arm's length value identified in other comparable companies.
\\
\\
The model, solved with LINGO software and particularly with its linear programming solver will be subject to weight sensitivity analysis, then an Analitycal Hierarchy Process will be used to avoid any incoherence and to assess in a precise way the weights to be assigned to each objective.
\pagebreak
\printbibliography
\end{document}
