\documentclass{article}

\usepackage{amsmath}
\usepackage{mathtools}
\usepackage{amstext}
\usepackage{cases}
\usepackage[colorinlistoftodos]{todonotes}
\usepackage[long]{optidef}
\usepackage[backend=bibier,
		url=false,
		doi=false,
		isbn=false,
		issn=false]{biblatex}
\addbibresource{ch1-2.bib}

\begin{document}

\title{Tax planning and BEPS: a transportation model for optimal profit allocation}

%\author{Marco Repetto, Davide La Torre}
%\date{March 2018}
\maketitle

\begin{center}
Marco Repetto\footnote{Department of Economics, Management and Quantitative Methods, University of Milan, Italy},
Davide La Torre\footnote{Dubai Business School, University of Dubai, UAE and Department of Economics, Management and Quantitative Methods, University of Milan, Italy.}
\end{center}

The paper deals with the problem of multinational entity allocation based on three main factors, namely the costs, the tax pressure and the functional characterization of a certain entity. The solution suggested is a Weighted Goal Programming model (WGP) that embeds these limitations imposed by the environment and tries to find an optimal solution that permits the Decision Maker (DM) to be competitive \textit{vis-à-vis} multinational competitors, avoiding any aggressive tax planning policy\cite{feller_three_2017}. The results obtained from such model are many times different from the ones put into places by multinationals, which tend to focus on minimizing only the tax base without taking into account other information that may result in significant value added from a tax planning perspective and at the same time lower the tax risk.
The way to avoid aggressive tax planning, is to take into account any possible index of potential tax risk, thanks to the work of OECD such measures were developed on the new Transfer Pricing guidelines \cite{TPguide2017}, however, this measure has to be considered in a wider perspective with the peculiar features of the firm considered.
Therefore choice was to pick the Multi-Criteria Decision Analysis; MCDA may be defined as a problem of multiple-objective programming, that differs from a linear one since more objective functions are handled; its formulation is the following:
	$$
	Min[f_1(x),f_2(x),...,f_k(x)] \quad i=1,...,k \quad where \quad k\geq2
	$$
	This approach seems to fit better the real world since, in reality, more than one objective is pursued, and most of the time in contrast with one another\cite{greco_multiple_2016}.
A solution to a multi-criteria problem would be optimal if it'd respected the Pareto Efficient assumption, namely that no other feasible solution exists that is at least as good with respect to all objectives and strictly better with respect to at least one objective. However such increase in complexity given by multiple objectives is both the strength and the weakness of such methodology, this is due to the fact that we have to deal with $N$ trade-offs deriving from the objectives we decided to include in our optimization problem. Because of this problem a lot of approaches and specific intelligent algorithms were proposed\cite{Cui2017}.
In the case under scope there's the necessity to controll completely the trade offs of the DM by controlling the importance given to each and every goal set by the model. Because of this necessity the ultimate choice was the Goal Programming Approach\cite{charnes_optimal_1955}.
The peculiarity of GP is given by its simplicity and it's flexibility\cite{colapinto_multi-criteria_2017}\cite{tamiz_goal_1998}; the idea behind such method is based on distance minimization. Since the goal of the model proposed is to account for direct trade-offs between all unwanted deviational variables the choice was to using weights which are not set a priori.The resulting mathematical formulation is the following one:

\begin{equation*}
\begin{aligned}
& \underset{n,p}{\text{minimize}}
& & \sum_{q=1}^{Q}(\frac{u_q n_q}{k_q}+\frac{v_q n_q}{k_q}) \\
& \text{subject to}
& & f_q(x)+n_q-p_q=b_q, \; q=1,...Q \\
& & & x\in F \\
& & & n_q,p_q\geq 0, \; q=1,...,Q
\end{aligned}
\end{equation*}

The Weighted Goal Programming was used extensively in the field of accounting\cite{aouni_goal_2017}. However, not every micro-field of accounting gained the same attention by both researchers and practitioner and, aside from its popularity, the interest in international tax planning never arose\cite{merville_transfer_1978}. This may be due to the typical background that such researchers have on the field. In fact, most of the researches moved in this field pertains to the legal area, which probably lacks from a point of view of mathematical modeling skills.
Therefore, seems necessary to propose a model which takes into account both new developments on OECD regulations as well as technological developments in hands of the Management.
The model proposed uses three different sources of data, such data source are: (i)macroeconomic, is the set of data used by DM that came from the national economies; (ii)target enterprise, are the types of data that refer particularly to the firm under scope; (iii)global enterprises data, these data come from other enterprises that are in some way similar to the enterprise under scope, both in terms of business or because of the functions they perform.
Given this set of data the model will have an objective, the minimization of the production costs, at the same time another economical objective will be the minimization of the tax liability defined as Earnings Before Interest and Taxes (EBIT) for the country corporate tax rate, then follows the soft economic objectives that are profit split maintenance in order to avoid the firm to lose its functional characterization and such functional characterization will have to lie within the arm's length value identified in other comparable companies.
The model, solved with LINGO software and particularly with its linear programming solver will be subject to weight sensitivity analysis, then an Analitycal Hierarchy Process\cite{Saaty1980} will be used to avoid any incoherence and to assess in a precise way the weights to be assigned to each objective.
\pagebreak
\printbibliography
\end{document}
