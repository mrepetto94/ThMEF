%\chapter{Comportamento illecito}\label{sec:AttaccanteIntrusore}
\chapter{Il Comportamento di un intrusore}\label{sec:AttaccanteIntrusore}
\begin{flushright}\begin{small}\textit{Volete nuocere a qualcuno? Non ditene male, ditene troppo bene.}\\
Andr� Siegfried\\
\end{small}\end{flushright}

\section{Attaccante - Intrusore}\label{sec:AttaccanteIntrusore2}
\begin{doublespace}
%Con l'esperessione
%Con l'espressione \emph{comportamento di un intrusore}\index{comportamento di un intrusore}, nell'ambito della sicurezza delle reti, intendiamo una associazione tra una persona ed un determinato suo modo di agire. S\\
Con il termine intrusore inten molti termini equivalenti:\cite{SCAP}, \cite{caselli}
%Questo suo modo di agire, pu� essere descritto in vari modi:
\begin{enumerate}
	%\item Man in the middle\index{Man in the middle},
	\item Attacker\index{Attacker},
	\item Intruder\index{Intruder},
	\item Cracker\index{Cracker},
	\item \dots
\end{enumerate}

\noindent Ne ho riportati solo alcuni in quanto la lista potrebbe essere davvero ampia. Tuttavia in questa tesi utilizzeremo il t

 pi� facile e standard\index{standard} trovare, sono reti composte da varie zone ben distinte tra loro, DMZ\index{DMZ}, zona demilitarizzata, per i servizi pubblici, una rete perimetrale, e infine una rete ``protetta\index{protetta}'' interna. Lo scenario appena presentato � un vise in zone.\\
Una questione molto spinosa ma al tempo stesso molto rilevante dal nostro punto di vista, � il concetto di \textbf{livello}.\\
Esistono infatti moltissimi studi che trattano il comportamento dell'attaccante come un atteggiamento casuale, randomico; il 

Pi� si vuole rappresentare la realt�\index{realt�} e pi� le categorie aumentano tuttavia rimane sempre impossibile definire una misura generale da attribuire anche  particolare contesto. In un ambito generale dove si pu� avere non solo un attaccante ma moltissimi altri come nel caso di Denial of service (DOS\index{DOS}) oppure Distribuited Dos; come valutare la distribuzione\index{distribuzione} se non in modo esponenziale\index{esponenziale}?.


\section{Motivazioni di un attacco}
Qual � o quali sono le motivazioni che spingono una persona a commettere un illecito\index{illecito}? Le risposte potrebbero essere molte e si potrebbero fare molti studi finalizzati alla loro scoperta.\\
si � cercato di studiare questo fenomeno e darne una possibile risposta. L'esperimento\index{esperimento} ha prodotto dei dati i quali hanno mostrato che un ane delle vulnerabilit� di un sistema o di un programma; quest'ultimo aspetto se risolto, porterebbe all'assenza del problema stesso; infatti � noto che le vulnerabilit� maggiori stanno nei software . Tuttavia non si potr� mai avere la certezza assoluta che del codice sia completamente esente da errori.\\

\end{doublespace}