\clearpage{\pagestyle{empty}\cleardoublepage}
\chapter*{Introduzione}% \label{introduzione}
\markboth{Introduzione}{Introduzione}
\addcontentsline{toc}{chapter}{Introduzione}

%%% Per l'epigrafe  %%%

\begin{flushright}\begin{small}\textit{"Le belle costruzioni sono pi� che scientifiche. Sono dei veri organismi, concepiti\\
spiritualmente; opere d'arte che si ispirano alla pi� raffinata tecnologia\\
e non alle idiosincrasie di un gusto modesto o ai compromessi con il committente."}\\
Frank Lloyd Wright\\
\end{small}\end{flushright}
%\newenvironment{linea}{\linespread{1.6}}

\begin{doublespace}
\indent Sempre pi� spesso la sicurezza informatica si intreccia con svariate discipline\index{discipline}, tra queste, quella economica\index{economica} offre molti strumenti di analisi\index{analisi} significativi. L'applicazione di metodologie\index{metodologie} di analisi economica agli aspetti di sicurezza informatica rappresenta l'ambito di ricerca\index{ricerca} innovativo nel quale � stata sviluppata questa tesi.\\
\indent Alcune delle domande per le quali si cerca di trovare risposte sono: come � possibile quantificare la sicurezza informatica? Come sviluppare modelli analitici\index{modelli analitici} adeguati? Quali conseguenze possono avere analisi economiche rispetto alle\\
\indent Valutazioni simili a quelle esposte vengono normalmente sviluppate sulla base di considerazioni qualitative delle caratteristiche di sicurezza dei sistemi.\\
\indent Obiettivo di questo nuovo ambito di ricerca � invece quello di fornire metriche\index{metriche} e stime quantitative\index{stime quantitative}, adatte quindi ad essere utilizzate in analisi comparative e simulazioni\index{simulazioni}.

\indent La tesi, in particolare, si � concentrata soprattutto sul problema della definizione di un modello analitico dell'attivit� di un attaccante; il quale prende in considerazione diversi pattern di comportamento che verranno definiti come danger, warning, loser, a seconda del loro grado di abilit�, questi pattern, verranno poi applicati in un contesto locale e successivamente in un ambito di rete.\\
\indent A questo scopo viene proposto un possibile metodo di profilazione della tipologia di attaccante e, tramite l'applicazione di tecniche basate sulle Catene di Markov\index{Markov}, ne viene modellata l'attivit� nel progredire di un'intrusione in una rete aziendale\index{rete aziendale}.\\
\indent Le catenerca.\\
\indent La tesi si apre attraverso il capitolo primo con una sostanziale introduzione dello stato dell'arte nel campo della dependability, e della sicurezza informatica rivolta allo studio del comportamento di attaccanti tramite ricerche quantitative. Il secondo capitolo, introduce le possibili soluzioni e problematiche relative allo sviluppo di un modello analitico per comprendere il comportamento di un attaccate attraverso un'analisi quantitativa riferito ad un ambito locale. Nel terzo capitolo viene introdotto il modello di Markov nel campo discreto. Il quarto capitolo, sfrutta i concetti esposti nel terzo per estendere la ricerca a dei modelli in ambito di rete. Nel quinto capitolo, si cerca di applicare i modelli di Markov studiati attraverso dei dati simulati e all'ausiglio di una rete di riferimento. Infine nel capitolo sesto, sono state riportate le conclusioni del lavoro ed eventuali sviluppi futuri.\\
\indent Sono state aggiunte anche alcune appendici, relative a dei concetti di probabilit� e statistica, questioni relative alla legislazione in ambito informatico, ed i dati delle simulazioni.

\end{doublespace}














%\begin{doublespace}
%Sempre pi� spesso la sicurezza informatica si intreccia con svariate discipline, tra questa, quella economica, rappresenta forse quel collegamento molto difficile da trattare.\\
%Come � possibile quantificare la sicurezza informatica?. Che modelli sono stati proposti, che soluzioni sono state adottate.\\
%In genere la sicurezza informatica, viene vista pi� come un aspetto qualitativo piuttosto che quantitativo.\\
%Un esempio sulla qualit� della sicurezza di un sistema, pu� essere la sua capacit� di resistere ad una variegata tipologia di attacchi, oppure al fatto di rendere il pi� difficile possibile l'acquisizione di informazione da parte di uno o pi� possibili attaccanti.\\
%Mentre l'aspetto quantitativo applicato alla sicurezza informatica , (sottolineo questo fatto in quanto esistono molti studi quantitativi sull'affidabilit� ``Reliability'' dei sistemi) spesso, non viene preso in considerazione; per fare un esempio; quanti firewall servono ad una banca per proteggersi da una certa tipologia di attaccanti?.\\
%Se assumiamo come buona l'idea che un attacco venga modellato tramite l'uso di funzioni esponenziali quindi stocastiche, � altrettanto vero che l'intrusore poi, nei successivi attacchi alla stessa macchina, si comporter� ancora in modo aleatorio?. Oppure � possibile cercare di dare una modellazione diversa da quella stocastica.\\
%\indent Noi ci concentreremo soprattutto sulla questione della tipologia degli attaccanti; cio� se sia possibile definire dei grafici che illustrino l'attivit� dell'attaccante in relazione allo sforzo, all'analisi dei costi/benefici e cercheremo inoltre di dire se dopo un attacco andato a buon fine, vi sia una distribuzione degli attacchi successivi alla stessa macchina.\\
%La successiva analisi su cui ci concentreremo sar� quella di capire se l'analogia tra affidabilit� e sicurezza sia davvero cos� forte; � davvero possibile considerare un attacco del tutto in modo stocastico, considerandolo come un problema di affidabilit�; faccio riferimento per esempio a \cite{Littlewood}.\\
%Infine, che cosa porta un intrusore a fermarsi?; quali sono le variabili da modellare per cercare di capire questo comportamento?.
%\end{doublespace}
\clearpage{\pagestyle{empty}\cleardoublepage}