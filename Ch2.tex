\documentclass{article}

\begin{document}

\title{Chapter 2}

\author{M. Repetto}

\date{\today}

\maketitle

\begin{abstract}
What we are going to do?

\end{abstract}

\section{Introduction}
Global Supply Chain Management is probably one of the most used terms when we talk about how the firms are running their business nowadays. GSCM may be defined as the allocation of goods and services along a series of transnational companies's global network to maximize profits and minimize waste. Inside this very wide concept we can find the concept of logistics which is in charge of the movement of goods, service and last but not least information from the sourcing of raw material, till it reaches the end costumer.
Along with theese two concepts a third one sticked with them, the concept of Green Supply Chain (GSC). This concept brougth into light by a more adavanced concern about environmental matters of the western countries forced the firms to be accountable for their negative externalities related to the evirnoment in which they operate. However such legislations lack from a point of view of legal contraints (setting only few qualitative restriction, poorly measurable) leaving some dregrees of freedom to the firms, is also important to notice the trend that is affecting such legislations, a trend that in the future may require firms to set particular frameworks to be accountable for their enviromental impact.

Because of that we propose a Goal Programming model in order to address such problem, following what proposed by ... we try to enhanche suh model fixing quantitative constraints to the pollution generated by the production activity invoveld in the creation of a good in our case a a networking electronic component (i.e. hub, switch or ruter). In order to measure such impact we'll use the framework used by Activity Based Costing in order to asses the marginal environmental impact of any additional unit elaborarated by the trans national firms in order to ultimately market the product on its reference market.

\section{Green Supply Chain}
Green Supply Chain may be defined as the series of interconnected activities across the border of different enterprises that adds value to the goods and services from the sourcing to the market. Wheresas Supply Chain Management sets its objective to maximize profits and minimize waste Green Supply Chain sets its objectives even further, posing has its ultimate mission to lower the ecological impact that a firm or a series of them has in their day to day operations. Such operation may involve:

-Green manufacturing and remanufactoring: 
-Green design: ;and 
-Green operations in general:

It's worth noting that even thoug CEO an Firms manager are looking for greener supply chain this doesn't mean that such interest was created by any increase in corporate social responsibility but 
\section{A state of the art review}

\section{The case study}
\subsection{The model}
\subsection{The results}

\section{Conclusion}



